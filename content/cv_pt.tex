%% start of file `template.tex'.
%% Copyright 2006-2013 Xavier Danaux (xdanaux@gmail.com).
%
% This work may be distributed and/or modified under the
% conditions of the LaTeX Project Public License version 1.3c,
% available at http://www.latex-project.org/lppl/.


\documentclass[11pt,a4paper,sans]{moderncv}        % possible options include font size ('10pt', '11pt' and '12pt'), paper size ('a4paper', 'letterpaper', 'a5paper', 'legalpaper', 'executivepaper' and 'landscape') and font family ('sans' and 'roman')

% moderncv themes
\moderncvstyle{casual}                             % style options are 'casual' (default), 'classic', 'oldstyle' and 'banking'
\moderncvcolor{orange}                               % color options 'blue' (default), 'orange', 'green', 'red', 'purple', 'grey' and 'black'
%\renewcommand{\familydefault}{\sfdefault}         % to set the default font; use '\sfdefault' for the default sans serif font, '\rmdefault' for the default roman one, or any tex font name
%\nopagenumbers{}                                  % uncomment to suppress automatic page numbering for CVs longer than one page

% character encoding
\usepackage[utf8]{inputenc}                       % if you are not using xelatex ou lualatex, replace by the encoding you are using
%\usepackage{CJKutf8}                              % if you need to use CJK to typeset your resume in Chinese, Japanese or Korean

% adjust the page margins
\usepackage[scale=0.75]{geometry}
%\setlength{\hintscolumnwidth}{3cm}                % if you want to change the width of the column with the dates
%\setlength{\makecvtitlenamewidth}{10cm}           % for the 'classic' style, if you want to force the width allocated to your name and avoid line breaks. be careful though, the length is normally calculated to avoid any overlap with your personal info; use this at your own typographical risks...

% personal data
\name{João}{Fernandes}
\title{Engenheiro Informático}                               % optional, remove / comment the line if not wanted
\address{Avenida 5 de Outubro 194 2º Direito}{1050-054 Lisboa}{Portugal}% optional, remove / comment the line if not wanted; the "postcode city" and and "country" arguments can be omitted or provided empty
\phone[mobile]{+351 917456174}                   % optional, remove / comment the line if not wanted
%\phone[fixed]{+2~(345)~678~901}                    % optional, remove / comment the line if not wanted
%\phone[fax]{+3~(456)~789~012}                      % optional, remove / comment the line if not wanted
\email{hi@joaofernandes.me}                               % optional, remove / comment the line if not wanted
\homepage{http://www.joaofernandes.me/}                         % optional, remove / comment the line if not wanted
%\extrainfo{I turn caffeine into code}                 % optional, remove / comment the line if not wanted
%\photo[64pt][0.4pt]{picture}                       % optional, remove / comment the line if not wanted; '64pt' is the height the picture must be resized to, 0.4pt is the thickness of the frame around it (put it to 0pt for no frame) and 'picture' is the name of the picture file
%\quote{Some quote}                                 % optional, remove / comment the line if not wanted

% to show numerical labels in the bibliography (default is to show no labels); only useful if you make citations in your resume
%\makeatletter
%\renewcommand*{\bibliographyitemlabel}{\@biblabel{\arabic{enumiv}}}
%\makeatother
%\renewcommand*{\bibliographyitemlabel}{[\arabic{enumiv}]}% CONSIDER REPLACING THE ABOVE BY THIS

% bibliography with mutiple entries
%\usepackage{multibib}
%\newcites{book,misc}{{Books},{Others}}
%----------------------------------------------------------------------------------
%            content
%----------------------------------------------------------------------------------
\begin{document}
%\begin{CJK*}{UTF8}{gbsn}                          % to typeset your resume in Chinese using CJK
%-----       resume       ---------------------------------------------------------
\makecvtitle

\section{Educação}
\cventry{2009--2014}{Mestrado Integrado em Engenharia Informática e Computação}{Faculdade de Engenharia da Universidade do Porto(FEUP)}{}{Média final: \textit{16/20}}{}  % arguments 3 to 6 can be left empty
\cventry{Set.2013-Fev.2014}{Aluno de Erasmus}{Sapienza Università di Roma}{}{Média: \textit{28,25/30}}{}

\section{Tese de Mestrado}
\cvitem{Título}{\emph{No-Touch Interfaces}}
\cvitem{Descrição}{Esta tese teve por objectivo definir um conjunto de regras de UX para interfaces No-Touch e criar uma framework para implementar este tipo de interfaces para web, utilizando o Microsoft Kinect. A aplicação utilizada como caso de estudo é uma aplicação para os médicos consultarem o historial clínico do paciente durante uma cirurgia.}
\cvitem{Nota final}{18/20}

\section{Experiência}
\cventry{Agosto 2014--Actualidade}{Engenheiro Informático}{Seedrs}{Lisboa}{}{Neste trabalho estou envolvido em desenvolvimento full-stack no website da Seedrs. Trabalho num ambiente agile com entrega continua. A nível de tecnologia, uso Ruby on Rails para backend, bem como Ember.js e AngularJS para outros projectos internos da Seedrs.}
\cventry{Fevereiro 2014--Julho 2014}{Estágio para tese de mestrado}{Glintt Healthcare Solutions}{Porto}{}{Implementação de uma framework para a criação de interfaces web utilizáveis com o Microsoft Kinect, usando uma aplicação existente como caso de estudo.}
\cventry{Março 2013--Junho 2013}{Monitor de Concepção e Análise de Algoritmos}{FEUP}{Porto}{}{Responsável por ajudar os alunos com os seus projectos e manter uma API em C++ para visualização de grafos.}
\cventry{Novembro 2012--Março 2013}{Responsável pelo programa}{Encontro Nacional de Estudantes de Informática 2013}{Porto}{}{Responsável por selecionar, contactar e receber oradores para o evento. O evento teve mais de 550 participantes, superando o recorde anterior de 300.}
%\cventry{year--year}{Job title}{Employer}{City}{}{Description}

\section{Competências}

\subsection{Experiência profissional}
\cvitem{Linguagens \& Frameworks}{Ruby on Rails, CSS, HTML, Javascript, JQuery, Ember.js, AngularJS, RSpec, Cucumber, mySQL}

\cvitem{Ferramentas \& Prácticas}{UNIX, git, nginx, Scrum, Testes de integração}

\subsection{Experiência académica/hobby}
\cvitem{Linguagens \& Frameworks}{C++, Java, C\#, C, JUnit, Microsoft Kinect, LeapMotion, Android, TinyOS, Scala, PostgreSQL}

\cvitem{Ferramentas \& Prácticas}{UML, Eclipse, Visual Studio, SOA}

\section{Linguagens}
\cvitemwithcomment{Português}{Língua Materna}{}
\cvitemwithcomment{Inglês}{Utilizador Avançado - Pontuação de 7.0 no IELTS}{}
\cvitemwithcomment{Castelhano}{Utilizador Intermédio}{}
\cvitemwithcomment{Italiano}{Utilizador Básico}{}

\section{Outros pontos}
\cvlistitem{Licensa de Primeiros Socorros - Suporte Básico de Vida emitido pela Kmed.}
\cvlistitem{Selecionado para apresentar um projecto de Inteligência Artificial a alunos do ensino secundário na "Semana Profissão Engenheiro".}
\cvlistitem{Pequena talk intitulada "UX in Natural User Interfaces" no SAPO Codebits.}
\cvlistitem{Curso de Princípios de Programação Funconal em Scala, no Coursera.}

% Publications from a BibTeX file without multibib
%  for numerical labels: \renewcommand{\bibliographyitemlabel}{\@biblabel{\arabic{enumiv}}}% CONSIDER MERGING WITH PREAMBLE PART
%  to redefine the heading string ("Publications"): \renewcommand{\refname}{Articles}
\nocite{*}
\bibliographystyle{plain}
\bibliography{publications}                        % 'publications' is the name of a BibTeX file

% Publications from a BibTeX file using the multibib package
%\section{Publications}
%\nocitebook{book1,book2}
%\bibliographystylebook{plain}
%\bibliographybook{publications}                   % 'publications' is the name of a BibTeX file
%\nocitemisc{misc1,misc2,misc3}
%\bibliographystylemisc{plain}
%\bibliographymisc{publications}                   % 'publications' is the name of a BibTeX file

\clearpage
\end{document}


%% end of file `template.tex'.
