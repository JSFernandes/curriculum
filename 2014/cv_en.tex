%% start of file `template.tex'.
%% Copyright 2006-2013 Xavier Danaux (xdanaux@gmail.com).
%
% This work may be distributed and/or modified under the
% conditions of the LaTeX Project Public License version 1.3c,
% available at http://www.latex-project.org/lppl/.


\documentclass[11pt,a4paper,sans]{moderncv}        % possible options include font size ('10pt', '11pt' and '12pt'), paper size ('a4paper', 'letterpaper', 'a5paper', 'legalpaper', 'executivepaper' and 'landscape') and font family ('sans' and 'roman')

% moderncv themes
\moderncvstyle{casual}                             % style options are 'casual' (default), 'classic', 'oldstyle' and 'banking'
\moderncvcolor{orange}                               % color options 'blue' (default), 'orange', 'green', 'red', 'purple', 'grey' and 'black'
%\renewcommand{\familydefault}{\sfdefault}         % to set the default font; use '\sfdefault' for the default sans serif font, '\rmdefault' for the default roman one, or any tex font name
%\nopagenumbers{}                                  % uncomment to suppress automatic page numbering for CVs longer than one page

% character encoding
\usepackage[utf8]{inputenc}                       % if you are not using xelatex ou lualatex, replace by the encoding you are using
%\usepackage{CJKutf8}                              % if you need to use CJK to typeset your resume in Chinese, Japanese or Korean

% adjust the page margins
\usepackage[scale=0.75]{geometry}
%\setlength{\hintscolumnwidth}{3cm}                % if you want to change the width of the column with the dates
%\setlength{\makecvtitlenamewidth}{10cm}           % for the 'classic' style, if you want to force the width allocated to your name and avoid line breaks. be careful though, the length is normally calculated to avoid any overlap with your personal info; use this at your own typographical risks...

% personal data
\name{João}{Fernandes}
\title{Software Engineer}                               % optional, remove / comment the line if not wanted
\address{Avenida 5 de Outubro 194 2º Direito}{1050-054 Lisboa}{Portugal}% optional, remove / comment the line if not wanted; the "postcode city" and and "country" arguments can be omitted or provided empty
\phone[mobile]{+351 917456174}                   % optional, remove / comment the line if not wanted
%\phone[fixed]{+2~(345)~678~901}                    % optional, remove / comment the line if not wanted
%\phone[fax]{+3~(456)~789~012}                      % optional, remove / comment the line if not wanted
\email{hi@joaofernandes.me}                               % optional, remove / comment the line if not wanted
\homepage{http://www.joaofernandes.me/}                         % optional, remove / comment the line if not wanted
\extrainfo{I turn caffeine into code}                 % optional, remove / comment the line if not wanted
%\photo[64pt][0.4pt]{picture}                       % optional, remove / comment the line if not wanted; '64pt' is the height the picture must be resized to, 0.4pt is the thickness of the frame around it (put it to 0pt for no frame) and 'picture' is the name of the picture file
%\quote{Some quote}                                 % optional, remove / comment the line if not wanted

% to show numerical labels in the bibliography (default is to show no labels); only useful if you make citations in your resume
%\makeatletter
%\renewcommand*{\bibliographyitemlabel}{\@biblabel{\arabic{enumiv}}}
%\makeatother
%\renewcommand*{\bibliographyitemlabel}{[\arabic{enumiv}]}% CONSIDER REPLACING THE ABOVE BY THIS

% bibliography with mutiple entries
%\usepackage{multibib}
%\newcites{book,misc}{{Books},{Others}}
%----------------------------------------------------------------------------------
%            content
%----------------------------------------------------------------------------------
\begin{document}
%\begin{CJK*}{UTF8}{gbsn}                          % to typeset your resume in Chinese using CJK
%-----       resume       ---------------------------------------------------------
\makecvtitle

\section{Education}
\cventry{2009--2014}{Master in Informatics and Computing Engineering}{Faculty of Engineering of the University of Porto(FEUP)}{}{Final average: \textit{16 out of 20}}{}  % arguments 3 to 6 can be left empty
\cventry{Sep.2013--Feb.2014}{Erasmus Student in Master in Informatics}{Sapienza Università di Roma}{}{Average: \textit{28,25/30}}{}
\cventry{2006--2009}{High School}{Escola Secundária da Maia}{}{Final average: \textit{17/20}}{}

\section{Master thesis}
\cvitem{Title}{\emph{No-Touch Interfaces}}
\cvitem{Description}{This thesis aimed to define a set of UX rules for No-Touch Interfaces and a framework to create such interfaces for the web, using the Microsoft Kinect as the technology of choice. The application used as a case study is an application for doctors to consult their patients clinical history during surgery.}
\cvitem{Final Grade}{18 out of 20}

\section{Experience}
\cventry{August 2014--Current}{Software Engineer}{Seedrs}{Lisbon}{}{In this role I am mainly involved in full-stack development of the Seedrs website. I work in an agile environment with continuous delivery. In terms of technology, I use Ruby on Rails for backend, and Ember.js and AngularJS for other projects within Seedrs.}
\cventry{February 2014--July 2014}{Intern for Master's Thesis}{Glintt Healthcare Solutions}{Porto}{}{Implemented a framework to build web interfaces usable with the Microsoft Kinect, using an existing application as a case study.}
\cventry{March 2013--June 2013}{Teaching Assistant for Algorithm Design and Analysis}{FEUP}{Porto}{}{Responsible for aiding students with their projects and maintaining a C++ API for visual graph representation.}
\cventry{November 2012--March 2013	}{Head of program}{National Meeeting of Informatics Students 2013}{Porto}{}{Responsible for selecting, contacting and welcoming speakers for the event. The event welcomed over 550 participants, breaking the previous record of 300.}
%\cventry{year--year}{Job title}{Employer}{City}{}{Description}

\clearpage

\section{Skills}

\subsection{Work experience}
\cvitem{Languages \& Frameworks}{Ruby on Rails, CSS, HTML, Javascript, JQuery, Ember.js, AngularJS, RSpec, Cucumber, mySQL}

\cvitem{Tools \& Practices}{UNIX, Git, nginx, Scrum, Integration testing}

\subsection{Academic/Hobby experience}
\cvitem{Languages \& Frameworks}{C++, Java, C\#, C, JUnit, Microsoft Kinect, LeapMotion, Android, TinyOS, Scala, PostgreSQL}

\cvitem{Tools \& Practices}{UML, Eclipse, Visual Studio, SOA}

\section{Languages}
\cvitemwithcomment{Portuguese}{Native Speaker}{}
\cvitemwithcomment{English}{Advanced User - IELTS Score of 7.0}{}
\cvitemwithcomment{Spanish}{Intermediate User}{}
\cvitemwithcomment{Italian}{Basic User}{}

\section{Miscellaneous}
\cvlistitem{First Aid - Basic Life Support license issued by Kmed.}
\cvlistitem{Had an Artificial Intelligence project selected for "Semana Profissão Engenheiro", where I had to present it to high school students and spark up their interest in software engineering.}
\cvlistitem{Attended SAPO Codebits thrice, and at the last edition I gave a small talk called "UX in Natural User Interfaces".}
\cvlistitem{Completed Functional Programming Principles in Scala course in Coursera.}


% Publications from a BibTeX file without multibib
%  for numerical labels: \renewcommand{\bibliographyitemlabel}{\@biblabel{\arabic{enumiv}}}% CONSIDER MERGING WITH PREAMBLE PART
%  to redefine the heading string ("Publications"): \renewcommand{\refname}{Articles}
\nocite{*}
\bibliographystyle{plain}
\bibliography{publications}                        % 'publications' is the name of a BibTeX file

% Publications from a BibTeX file using the multibib package
%\section{Publications}
%\nocitebook{book1,book2}
%\bibliographystylebook{plain}
%\bibliographybook{publications}                   % 'publications' is the name of a BibTeX file
%\nocitemisc{misc1,misc2,misc3}
%\bibliographystylemisc{plain}
%\bibliographymisc{publications}                   % 'publications' is the name of a BibTeX file

\clearpage
%-----       letter       ---------------------------------------------------------
% recipient data
%\recipient{Company Recruitment team}{Company, Inc.\\123 somestreet\\some city}
%\date{January 01, 1984}
%\opening{Dear Sir or Madam,}
%\closing{Yours faithfully,}
%\enclosure[Attached]{curriculum vit\ae{}}          % use an optional argument to use a string other than "Enclosure", or redefine \enclname
%\makelettertitle

%Lorem ipsum dolor sit amet, consectetur adipiscing elit. Duis ullamcorper neque sit amet lectus facilisis sed luctus nisl iaculis. Vivamus at neque arcu, sed tempor quam. Curabitur pharetra tincidunt tincidunt. Morbi volutpat feugiat mauris, quis tempor neque vehicula volutpat. Duis tristique justo vel massa fermentum accumsan. Mauris ante elit, feugiat vestibulum tempor eget, eleifend ac ipsum. Donec scelerisque lobortis ipsum eu vestibulum. Pellentesque vel massa at felis accumsan rhoncus.

%Suspendisse commodo, massa eu congue tincidunt, elit mauris pellentesque orci, cursus tempor odio nisl euismod augue. Aliquam adipiscing nibh ut odio sodales et pulvinar tortor laoreet. Mauris a accumsan ligula. Class aptent taciti sociosqu ad litora torquent per conubia nostra, per inceptos himenaeos. Suspendisse vulputate sem vehicula ipsum varius nec tempus dui dapibus. Phasellus et est urna, ut auctor erat. Sed tincidunt odio id odio aliquam mattis. Donec sapien nulla, feugiat eget adipiscing sit amet, lacinia ut dolor. Phasellus tincidunt, leo a fringilla consectetur, felis diam aliquam urna, vitae aliquet lectus orci nec velit. Vivamus dapibus varius blandit.

%Duis sit amet magna ante, at sodales diam. Aenean consectetur porta risus et sagittis. Ut interdum, enim varius pellentesque tincidunt, magna libero sodales tortor, ut fermentum nunc metus a ante. Vivamus odio leo, tincidunt eu luctus ut, sollicitudin sit amet metus. Nunc sed orci lectus. Ut sodales magna sed velit volutpat sit amet pulvinar diam venenatis.

%Albert Einstein discovered that $e=mc^2$ in 1905.

%\[ e=\lim_{n \to \infty} \left(1+\frac{1}{n}\right)^n \]

%\makeletterclosing

%\clearpage\end{CJK*}                              % if you are typesetting your resume in Chinese using CJK; the \clearpage is required for fancyhdr to work correctly with CJK, though it kills the page numbering by making \lastpage undefined
\end{document}


%% end of file `template.tex'.
